% Commands for creating cards

\setlength{\parindent}{0em}
\setlength{\parskip}{1em}
\renewcommand{\baselinestretch}{0.75}

\pgfmathsetmacro{\cardwidth}{6.3}
\pgfmathsetmacro{\cardheight}{8.8}
\pgfmathsetmacro{\imagewidth}{\cardwidth}
\pgfmathsetmacro{\imageheight}{0.75*\cardheight}
\pgfmathsetmacro{\stripwidth}{0.7}
\pgfmathsetmacro{\strippadding}{0.2}
\pgfmathsetmacro{\textpadding}{0.1}
\pgfmathsetmacro{\titley}{\cardheight-0.5}
\pgfmathsetmacro{\DescriptionWidth}{(\cardwidth-2*\strippadding-2*\textpadding)*1cm}
\pgfmathsetmacro{\DescriptionTextWidth}{(\cardwidth-2*\strippadding-\stripwidth-\textpadding)*1cm}


\def\shapeCard{
	(0,0)
    rectangle (\cardwidth, \cardheight)
}

\def\shapeLeftStripShort{
	(\strippadding, \cardheight+1)
    rectangle (\strippadding+\stripwidth, \cardheight-1)
}

\def\shapeLeftStripShortBelow{
	(\strippadding, \cardheight+1)
    rectangle (\strippadding+\stripwidth, \cardheight-1.5)
}

\def\shapeRightStripShort{
	(\cardwidth-\stripwidth-\strippadding,\cardheight+1)
    rectangle (\cardwidth-\strippadding, \cardheight-1)
}

\def\shapeTitleArea{
	(2*\strippadding+\stripwidth, \cardheight)
    rectangle (\cardwidth-2*\strippadding-\stripwidth, \cardheight-1)
}

\def\shapeContentArea{
	(\strippadding, 0.5*\cardheight)
    rectangle (\cardwidth-\strippadding, \strippadding)
}

\def\shapeSetBuble{
	(\cardwidth-\strippadding-0.1, \strippadding+0.1)
    circle (0.35)
}

\tikzset{
    cardcorners/.style={
        rounded corners=0.2cm
    },
    elementcorners/.style={
        rounded corners=0.1cm
    },
    stripshadow/.style={
        drop shadow={
            opacity=.5,
            color=black
        }
    },
    strip/.style={
        elementcorners,
    },
    cardimage/.style={
        path picture={
            \node[below=-1.5mm] at (0.5*\cardwidth,\cardheight) {
                \includegraphics[width=\imagewidth cm]{#1}
            };
        }
    },
}

% Card
\newcommand{\Card}[2]{
	\unitlength=1mm
    \begin{tikzpicture}[x=1cm, y=1cm]
    	\draw[cardcorners, cardimage=#1] \shapeCard;
    	#2
    	\draw[lightgray, cardcorners] \shapeCard;
    \end{tikzpicture}
}

\newcommand{\carddebug}{
    \draw [step=1,help lines] (0,0) grid (\cardwidth,\cardheight);
}

% Cost
\newcommand{\CardCost}[1]{
    \begin{scope}[even odd rule]
        \clip[cardcorners] \shapeCard;
        \fill[strip, CostBgColor] \shapeRightStripShort;
    \end{scope}
    \node at (\cardwidth-0.5*\stripwidth-\strippadding, \titley-0.1) {
    	\color{black}
        #1
    };
}

% Power
\newcommand{\CardPower}[1]{
    \begin{scope}[even odd rule]
        \clip[cardcorners] \shapeCard;
        \fill[strip, CostBgColor] \shapeRightStripShort;
    \end{scope}
    \node at (\cardwidth-0.5*\stripwidth-\strippadding, \titley-0.1) {
    	\color{black}
        #1
    };
}

% Starting Equipment
\newcommand{\StartingEquipment}{
	\begin{scope}[even odd rule]
        \clip[cardcorners] \shapeCard;
        \fill[strip, black] \shapeLeftStripShortBelow;
    \end{scope}
    \node at (0.5*\stripwidth+\strippadding, \titley-0.75) {
    	\color{white}
        \small
        \faUser
    };
}

% Set
\newcommand{\CardSet}[1]{
    \begin{scope}[even odd rule]
        \clip[cardcorners] \shapeCard;
        \fill[strip, gray] \shapeSetBuble;
    \end{scope}
    \node at (\cardwidth-\strippadding-0.1, \strippadding+0.1) {
    	\color{white}
        #1
    };
}

% Cost
\newcommand{\CardType}[2]{
    \begin{scope}[even odd rule]
        \clip[cardcorners] \shapeCard;
        \fill[strip, #1] \shapeLeftStripShort;
    \end{scope}
    \node at (0.5*\stripwidth+\strippadding, \titley-0.1) {
        #2
    };
}

% Title
\newcommand{\CardTitle}[1]{
    \fill[elementcorners, TitleBgColor, opacity=.85] \shapeTitleArea;
    \node[text width=3.75cm] at (0.5*\cardwidth,\titley) {
        \begin{center}
            \color{white}\normalsize #1
        \end{center}
    };
}

% Content
\newcommand{\CardContent}[1]{
    \begin{scope}[even odd rule]
        \clip[cardcorners] \shapeCard;
        \fill[elementcorners, ContentBgColor] \shapeContentArea;
    \end{scope}
    \node[below right, text width=\DescriptionWidth]
    at (\strippadding+\textpadding, 0.5*\cardheight-\textpadding) {
        {
        	\scriptsize
            \par
            #1
        }
    };
}

% Content
\newcommand{\CardFlavorText}[1]{
    \node[below right, text width=\DescriptionWidth]
    at (\strippadding + \textpadding, 1.5) {
        {
        	\tiny
            \par
            #1
        }
    };
}

% Ability
\newcommand{\Ability}[3] {
	\parbox{4mm}{
    	{\normalsize #1 \hfill}
    }
	\parbox{50mm}{
    	#2
        {\tiny #3}
    } \\[2mm]
}

% Repeat
\newcommand{\Repeat}[2]{
    \foreach \n in {1,...,#1}{#2}
}

