\newcommand{\CycleAction}[1]{
    \Ability
    	{\faRefresh}
    	{Re-equip #1}
        {(Discard up to #1 cards to draw that many cards)}
}
\newcommand{\AttackAction}[1]{
    \Ability
    	{\faCrosshairs}
    	{Attack: #1}
        {(Adds #1 to a mercenaries combat strength)}
}
\newcommand{\ConsumeAction}[1]{
    \Ability
    	{\faFlask}
    	{Consume: #1}
        {(Spend this card to use this ability)}
}
\newcommand{\DrawAction}[1]{
    \Ability
    	{\faShareSquareO}
    	{Draw #1}
        {(Draw #1 cards from your deck)}
}
\newcommand{\CombatAction}[1]{
    \Ability
    	{\faShield}
    	{Combat #1}
        {(Combat strength of #1)}
}
\newcommand{\LootAction}[2]{
    \Ability
    	{\faShoppingCart}
    	{Loot #1/#2}
        {(Look at the top #1 cards of the hall stack stack and keep #2)}
}
\newcommand{\StealAction}[1]{
    \Ability
    	{\faLowVision}
    	{Steal #1}
        {(You may take the bottom card from the hall stack)}
}
\newcommand{\GuideAction}[1]{
    \Ability
    	{\faMapO}
    	{Guide #1}
        {(You may skip up to the first #1 cards from any area stack when looting)}
}
\newcommand{\BankrollAction}[1]{
    \Ability
    	{\faCreditCard}
    	{Bankroll #1}
        {(Produces #1 gold)}
}
\newcommand{\DropAction}[1]{
    \Ability
    	{\faCartPlus}
    	{Drops #1}
        {(When defeated you may choose to acquire one copy of the listed card(s).)}
}
\newcommand{\BuyAction}[1]{
    \Ability
    	{\faShoppingBasket}
    	{Buy #1}
        {(You may purchase a spent card at a discount of #1, not to go below 1)}
}
\newcommand{\AutomaticAction}[1]{
    \Ability
    	{\faExclamation}
    	{Event: #1}
        {}
}

\newcommand{\DesertAction}[1]{
    \Ability
    	{\faChainBroken}
    	{Desertion: #1}
        {}
}
\newcommand{\StackAbility}[1]{
    \Ability
    	{\faClone}
    	{Stacking: #1}
    	{}
}
\newcommand{\InspireAbility}[1]{
    \Ability
    	{\faClone}
    	{Inspire: #1}
    	{}
}
\newcommand{\FortifyAction}[1]{
    \Ability
    	{\faShield}
    	{Fortify: #1}
        {}
}
\newcommand{\CancelAction}[1]{
    \Ability
    	{\faRemove}
    	{#1}
        {}
}
\newcommand{\KillAction}[1]{
    \Ability
    	{\faUserTimes}
    	{Kill: #1}
        {}
}