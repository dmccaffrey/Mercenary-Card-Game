{
\setlength{\parskip}{1em}
\titlespacing*{\section}{0pt}{0.5em}{0.5em}
\titlespacing*{\subsection}{0pt}{0.5em}{0em}
\titlespacing*{\subsubsection}{0pt}{0.5em}{0em}
\setlength{\columnsep}{8mm}
\setlength{\columnseprule}{0.1mm}
\noindent
{
   	\bf
    \color{red}
  	\Huge{Mercenary Game}
   	\hfill
   	\large{Notes}
}
\rule{\textwidth}{0.5mm}
\begin{multicols}{2}
\section{Setup}
\subsection{Starting Hand}
You start with 10 cards, four of which are predetermined (these are marked with the player logo).

\subsection{Dividing the map}
Starting with the first player, each player selects and marks a territory. Territories that have been marked by a player provide a 3 strength bonus for that player during any battle that occurs there.

\section{Play}
\subsection{Selecting a territory}
Before combat the player with priority selects a territory for battle. This can be any territory, including one that has already been conqered.

\subsection{Combat Turn}
Each player may play any number of cards each turn, or pass.

\subsection{End of Combat}
Once both players have passed, combat is concluded. The player with the highest total mercenary value wins the battle and secures the territory.

\subsection{End of turn}
At the end of the turn you may draw back up to 10, or 5 cards (whichever is lower).
\end{multicols}
}